\renewcommand{\baselinestretch}{1.0}
% How to insert a figure
\begin{figure}[htbp]
\centerline{
\tikzset{
    state/.style={
           rectangle,
           rounded corners,
           draw=black, very thick,
           anchor=south west,
           align=center
           },
}
    \begin{tikzpicture}
        \draw[thick, >->]    (-5mm,0mm)      --      (97.5mm,0mm);
        %
        \draw[thick]    (0mm,2.5mm)      --      (0mm,-2.5mm);
        \node[align=center, anchor=north] at (0mm,-5mm)
        {Week 12\\
        18/05/15};
        %
        \draw[thick]    (32.5mm,2.5mm)      --      (32.5mm,-2.5mm);
        \node[align=center, anchor=north] at (32.5mm,-5mm)
        {Week 13\\
        25/05/15};
        %
        \draw[thick]    (65mm,2.5mm)      --      (65mm,-2.5mm);
        \node[align=center, anchor=north] at (65mm,-5mm)
        {Semester Two\\
        20/07/15};
        %
        \node[state] at (-10mm,23.5mm) 
        {\textbf{Physical Model Specification}};
        \draw[thick, ->]    (0mm,23.5mm)     --       (0mm,3.5mm);
        %
        \node[state] at (22.5mm,16mm)
        {\textbf{Physical Model Design}};
        \draw[thick, ->]    (32.5mm,16mm)     --       (32.5mm,3.5mm);
        %
        \node[state] at (55mm,8.5mm) 
        {\textbf{Physical Proof of Concept I}};
        \draw[thick, ->]    (65mm,8.5mm)     --       (65mm,3.5mm);        %
    \end{tikzpicture}
}
\caption{Physical model specification and realisation phase.}
\label{time4}
\end{figure}
\renewcommand{\baselinestretch}{1.5}
